\section{Введение}

В современном цифровом обществе компьютерные вирусы стали одной из наиболее серьезных угроз для безопасности информации и непрерывности работы компьютерных систем. С каждым годом количество и разнообразие вирусов продолжают увеличиваться, и их воздействие становится все более разрушительным. От простых вирусов, наносящих ущерб файлам и программам, до сложных многокомпонентных малварей, способных перехватывать личную информацию и шифровать данные, компьютерные вирусы представляют собой серьезную угрозу как для индивидуальных пользователей, так и для организаций.

В данном реферате мы рассмотрим различные аспекты компьютерных вирусов, начиная с их определения и истории, и заканчивая методами защиты и превентивными мерами. Мы проанализируем различные виды вирусов, их характеристики и методы действия, а также рассмотрим последствия, которые они могут иметь для компьютерных систем и пользователей. В конечном итоге мы обсудим важность осведомленности о компьютерных угрозах и методов защиты, которые могут помочь минимизировать риски воздействия вирусов и обеспечить безопасность информации в цифровом мире.

Понимание природы и характеристик компьютерных вирусов является ключевым аспектом обеспечения кибербезопасности в нашей современной информационной эпохе. Давайте начнем наше исследование и углубимся в мир компьютерных вирусов, чтобы лучше защитить себя и свои данные от этой постоянно угрожающей опасности.


