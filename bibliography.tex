
\addcontentsline{toc}{section}{Список литературы}
 

\begin{thebibliography}{}
    \bibitem{litlink1}  Афанасьева Д.В.  - Компьютерные вирусы: специфика и противодействие // Наука, образование и культура. 2019. №. 3 (37). С. 11–12.
    \bibitem{litlink2}  Атамкулова М. Т., Саримсаков А. А. Компьютерные вирусы и антивирусные программы //Известия Ошского технологического университета. 2016. Т. 2. С. 136–140.
    \bibitem{litlink3}  Ганижева Н. Ж. Компьютерные вирусы и антивирусные программы //Молодой ученый. 2021. №. 33. С. 3–5.
    \bibitem{litlink4}  Зенкин Д. В., Касперский Е. В. Компьютерные вирусы: происхождение, реальная угроза и методы защиты, режим доступа: свободный / [Электронный ресурс] URL: https://www.nkj.ru/archive/articles/7889/ (дата обращения: 20.04.2024).
    \bibitem{litlink5} 	Попов Илья Олегович, Марунько Анна Сергеевна, Петров Олег Игревич, Олейник Анастасия Александровна Вирусы и антивирусные программы в информационной безопасности // Научные записки молодых исследователей. 2020. №4. URL: https://cyberleninka.ru/article/n/virusy-i-antivirusnye-programmy-v-informatsionnoy-bezopasnosti (дата обращения: 20.04.2024). 
    \bibitem{litlink6}  Козлов Захар Сергеевич КОМПЬЮТЕРНЫЕ ВИРУСЫ И АНТИВИРУСЫ // Столыпинский вестник. 2022. №4. URL: https://cyberleninka.ru/article/n/kompyuternye-virusy-i-antivirusy (дата обращения: 20.04.2024). 
\end{thebibliography}